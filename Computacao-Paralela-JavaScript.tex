\documentclass{article}

%% Better math support:
\usepackage{amsmath}

%% Bibliography style:
\usepackage{mathptmx}           % Use the Times font.
\usepackage{graphicx}           % Needed for including graphics.
\usepackage{url}                % Facility for activating URLs.

%% Set the paper size to be A4, with a 2cm margin
%% all around the page.
\usepackage[a4paper,margin=2cm]{geometry}

%% Natbib is a popular style for formatting references.
\usepackage{natbib}
%% bibpunct sets the punctuation used for formatting citations.
\bibpunct{(}{)}{;}{a}{,}{,}

%% textcomp provides extra control sequences for accessing text symbols:
\usepackage{textcomp}
\newcommand*{\micro}{\textmu}
%% Here, we define the \micro command to print a text "mu".
%% "\newcommand" returns an error if "\micro" is already defined.

%% This is an example of a new macro that I've created to save me
%% having to type \LaTeX each time.  The xspace command provides space
%% after the word LaTeX where appropriate.
\usepackage{xspace}
\providecommand*{\latex}{\LaTeX\xspace}
%% "\providecommand" does nothing if "\latex" is already defined.

\usepackage[portuguese]{babel}
\usepackage[utf8]{inputenc}
\usepackage[T1]{fontenc}

%%%%%%%%%%%%%%%%%%%%%%%%%%%%%%%%%%%%%%%%%%%%%%%%%%%%%%%%%%%%%%%%%%%%%%
%% Start of the document.
%%%%%%%%%%%%%%%%%%%%%%%%%%%%%%%%%%%%%%%%%%%%%%%%%%%%%%%%%%%%%%%%%%%%%%

\begin{document}

\author{Clauber Pereira Stipkovic Halic\\
  Universidade Mackenzie\\
  Orientador: Dr. Calebe de Paula Bianchini\\
  Apoio PIVIC Mackenzie}
\date{\today}
\title{COMPUTAÇÃO PARALELA EM JAVASCRIPT}
\maketitle

\section{RESUMO}

JavaScript é a linguagem de programação com maior porcentagem de uso na Web
atualmente, por conta da massiva utilização da linguagem em web sites e aplicações ou até
mesmo em sistemas operacionais como o Firefox OS da Mozilla, a demanda por rapidez e
eficiência de uso de recursos computacionais, faz com as JavaScript Engines sejam cada
vez mais exigidas em questão de performance, gerenciamento de memória e uso bateria, e
para isso, o interesse em aliar métodos de computação paralela com a linguagem tem
aumentando com o passar dos anos. Nesta iniciação científica, apresentamos e definimos a
linguagem JavaScript, o funcionamento de uma JavaScript Engine genérica mostrando quais
são os passos pelos quais um código JavaScript percorre até ser executado pelo browser,
dando ênfase no comportamento da JavaScript Engine da Fundação Mozilla, a
SpiderMonkey, atrelada com o modo de funcionamento do Just in Time Compiler.
Apresentamos o conceito de computação paralela utilizado atualmente para aplicar recursos
como o SIMD e que é implementado por um projeto que tem a finalidade de prover recursos
como tipos com os quais é possível utilizar computação paralela para aplicações web
através de bibliotecas JavaScript como a SIMD.js que atualmente é desenvolvida por
empresas como Mozilla, Google e Intel.
Palavras chave: JavaScript; Computação Paralela; SIMD.

\section{ABSTRACT}

JavaScript is a programming language with the highest percentage of use on the Web today,
due to the massive use of the language in websites and applications or even operating
systems such as Mozilla's Firefox OS, the demand for speed and resource to an efficiency
computer usage, causes an increasingly required of JavaScript Engines in matter of
performance, memory management and battery usage, and because of it, the interest in
combining methods of parallel computing with the language has increased over the years.
This scientific research, presents and defines the JavaScript language, the operation of a
generic JavaScript Engine showing what are the steps by which JavaScript code runs to be
executed by the browser, emphasizing the JavaScript Engine behavior of the Mozilla
Foundation SpiderMonkey, tied with the operation of the JIT Compiler. We are introducing the
concept of parallel computing currently used to implement features such as SIMD and which
is implemented by a project that aims to provide features such as types with which can be
use parallel computing to web applications through JavaScript libraries like SIMD.js which is
currently developed by companies like Mozilla, Google and Intel.
Keywords: JavaScript; Parallel Computing; SIMD.

\section{INTRODUÇÃO}

Nos últimos 20 anos, a World Wide Web (conhecida popularmente como Web) se tornou
uma ferramenta absolutamente importante para o mundo moderno de modo geral, mudando
o modo como o mundo se comunica, executa transações financeiras, aprende entre outras
atividades cotidianas, provendo produtos e serviços para suprir essas necessidades,
encurtando distancias e facilitando a vida dos usuários. Desde a chamada “guerra dos
navegadores”, em meados de 1995, entre os web browsers Microsoft Internet Explorer e
Netscape Navigator, quando houve a disputa por mercado entre os dois web browser, o foco
dos desenvolvedores de aplicações mudou consideravelmente para acompanhar a transição
do foco tipicamente desktop, para o foco multi-device.
No início dos anos 90, muitas aplicações eram desenvolvidas apenas e exclusivamente para
funcionarem nos desktops; porém, com o passar dos anos, a chegada dos web browsers, e
posteriormente, a evolução das linguagem web e dos devices portáteis, fez com que
desenvolvedores e empresa ligadas a tecnologia, passassem a ver a web como uma
plataforma com imenso potencial para que as aplicações já existentes no desktops, fossem
levadas para os web browser, e para que surgissem novas aplicações feitas inteiramente
pensando apenas na sua utilização através da web.
Com a popularização da internet, a facilidade para adquirir um computador e obter acesso a
web, fez com que muitos desenvolvedores e empresas surgissem focadas na criação de
aplicações que poderiam ser utilizadas somente pela web. Por volta do ano de 2005, surgiu
o conceito de Web 2.0, criado e difundido por Tim O'Reilly, que tinha como objetivo fomentar
definitivamente a web como plataforma de desenvolvimento de aplicações, o que abriu
espaço para o surgimento de muitos produtos e serviços como leitores de e-mail, aplicações
de agências bancarias, etc, que nascer voltados inteiramente para a web, e não mais
exclusivamente para desktops,
O surgimento de aplicações mais complexas, que utilizam massivamente a linguagem
JavaScript, e que demandam recursos computacionais, foi liderado pelo conceito da Web
2.0, o que deu início a preocupação com a velocidade e eficiência de execução de
aplicações por parte dos web browsers, tornando as tecnologias que dão suporte a essas
aplicações mais importantes com o passar do tempo.
Atualmente a linguagem JavaScript não mais executada exclusivamente em web browser,
mas também pode ser utilizada para criar aplicações que são executadas no lado servidor,
para a escrita de testes de aplicações, entre outras funcionalidades. Todo esse novo
ecossistema está disponível através do uso do framework NodeJS, que prove aos
desenvolvedores, acesso direto a JavaScript Engine do Google chamada V8. Alem disso,
temos esforços da Fundação Mozilla para levar o JavaScript para os dispositivos móveis e
smart TVs, criando um sistema operacional baseado inteiramente em tecnologias web com o
Firefox OS.

\section{A LINGUAGEM JAVASCRIPT}

A linguagem JavaScript foi criada em 1995 pelo desenvolvedor americano, Brendan Eich,
como parte do web browser Netscape Navigator, para que este tivesse a habilidade de
interpretar scripts do lado cliente, ou seja, os web browsers.
\begin{quote}
“JavaScript is a programming language that adds interactivity to your
website (e.g. games, responses when buttons are pressed or data entered in
forms, dynamic styling, animation).” (Mozilla Developer Network)
\end{quote}

David Flanagan (2011) define que JavaScript é uma linguagem de programação dinâmica e
multi-paradigma (script, orientada a objetos, imperativa e funcional), conhecida por ser
utilizada para interagir com o lado cliente (através da plataforma Document Object Model),
comunicação assíncrona e mais atualmente por também haver a possibilidade de ser
executada do lado servidor, através do projeto NodeJS.
Seu desenvolvimento teve como base a padronização da especificação ECMA-2623 e
ISO/IEC 16262. Ao final de sua implementação a Netscape (atualmente Fundação Mozilla)
submeteu em novembro de 1996 a especificação da linguagem JavaScript, que foi aceita e
teve a definição do padrão chamada de ECMAScript. A partir de 2012 os browsers mais
modernos como Mozilla Firefox, Google Chrome e Opera Browser passaram a dar suporte
completo ao ECMAScript 5.1. Em Junho 17 de Junho de 2015, a ECMAScript 6 foi publicada
oficialmente e é popularmente chamada pelos desenvolvedores de ES6.

\section{DEFINIÇÃO DE UMA JAVASCRIPT ENGINE}

Conforme definição apresentada pelo site Mozilla Developer Network, uma JavaScript
Engine é “A JavaScript Engine é um interpretador que faz o parsing e executa programas
JavaScript.” (tradução), e também é encontrada como JavaScript interpreter ou JavaScript
implementation, mas na essência uma JavaScript Engine é uma Virtual Machine (Maquina
Virtual).
A JavaScript Engine é construída seguindo como base o mesmo princípio de uma virtual
machine, e que contem passos que são comuns e definidos para todo tipo de virtual
machine:
\begin{itemize}
\item Parser
\item Intermediate Representation (IR)
\item Interpreter
\item Just-in-Time Compiler / Garbage Collection
\item Optimization
\end{itemize}
Originalmente, a primeira JavaScript Engine foi construída para possibilitar a execução de
código JavaScript no web browser Netscape Navigator, mas atualmente, é utilizada também
para execução de códigos JavaScript em CLI (Command Line Interface), o que possibilita
utilizar a linguagem também em client-side como ferramenta para automação de tarefas, e
até mesmo em aplicações mais robustas onde existe a necessidade de conexões
simultâneas com o servidor, onde é utilizado por exemplo, o protocolo WebSocket
implementado no framework NodeJS.

\subsection{FLUXO DE FUNCIONAMENTO DA JAVASCRIPT ENGINE}

O esquema teórico apresentado na Figura 1, mostra o fluxo que um código JavaScript
percorre na JavaScript Engine, rodando em um processador multi-core, desde recebimento
do código até a sua execução.

IMAGEM1

Partindo do início no caminho percorrido pelo código JavaScript, como mostrado no
esquema acima, utilizamos um trecho de código que faz uma soma entre dois números e
retorna o resultado dessa soma, como prova de conceito para mostrar passo a passo os
resultados de cada etapa do processo na JavaScript Engine. O trecho de código pode ser
visto na Figura 2.

IMAGEM2

Ao receber o trecho de código, a JavaScript Engine inicia o parsing (conhecido em
português como analise sintática), processo que consiste em analisar uma sequência de
símbolos ou caracteres em linguagem natural ou de computador, construindo uma estrutura
hierárquica correspondente ao que foi informado antes do método de parsing ser aplicado.
É durante o processo de parsing também que a validação sintática do código, procurando
por possíveis erros baseado na escrita padrão da linguagem é feito, para evitar erros
durante a compilação ou resultados divergentes na execução do código ao final do
processo.
Utilizado a função de soma como exemplo, o resultado do parsing utilizando o método
tokenizer\footnote{Tokenizer pode ser um programa que executa analises léxicas em uma sequencia de caracteres.}, temos o resultado do trecho de código da função soma que será gerado e é
apresentado na Figura 3:

IMAGEM3

Após a geração, uma Abtract Syntax Tree (AST) é formada, como no exemplo para a função
de soma conforme mostrado na Figura 4:

IMAGEM4

Uma vez que a Abstract Syntax Tree é formada, com base no tipo da JavaScript Engine, o
bytecode generator converte a Abstract Syntax Tree para uma linguagem intermediaria ou
código nativo\footnote{A etapa de geração do bytecode, pode variar nas diferentes implementações de JavaScript Engine, como
por exemplo a Rhino \url{https://developer.mozilla.org/en-US/docs/Mozilla/Projects/Rhino/JavaScript_Compiler},
que por ser escrita em linguagem Java, adiciona a etapa de tradução do código JavaScript para classes
Java.}
, e isso é feito para cada bloco de código dentro da Abstract Syntax Tree.
Bytecodes são formas canónicas de representação de códigos (definido no livro do dragão
como Intermediate Representation) que são projetados para obter execução eficiente por
software interpretador. Assim como muitos interpretadores, uma JavaScript Engine é uma
função simples, mas tremendamente longa, que percorre os bytecodes, uma instrução por
vez, usando a instrução switch (ou alternativas mais rápidas, dependendo do compilador)
para chegar até o ponto apropriado do código para a instrução que será executada.
Seguindo o passo de interpretação e compilação na Figura 5, podemos verificar o bytecode
gerado para a função “sum” que utilizamos como referência:

IMAGEM5

Após a geração do bytecode, entramos na fase de interpretação e execução da
representação intermediária, o que é comumente feito utilizando uma função simples, em
vários passos, uma instrução por vez percorrendo o bytecode gerado. Quando o bytecode
chega até a area de execução, observamos a existência de dois componentes importantes
para a performance de uma JS engine, que são o JIT (Just-in-Time Compiler) e o Garbage
Collector.
O JIT, é responsável por traduzir bytecodes para código de maquina durante a execução do
programa, ou seja, assim que um trecho de código é requisitado, o JIT transforma o
bytecode para código de maquina, referente ao bloco solicitado e o disponibiliza para
execução, fazendo com que a disponibilização do resultado seja rápida e não consuma
ciclos de processamento desnecessários entregando somente os blocos que serão
utilizados no momento em que forem requeridos.
Junto ao JIT, encontramos o Garbage Collector (ou coletor de lixo), que é a área
responsável pelo gerenciamento automático da memória que é ocupada por objetos, sendo
denominados como “lixo” as posições de memória que estão sendo ocupadas com
informações referentes ao programa em execução mas que não são mais relevantes ou que
não são mais utilizadas.
Tanto o JIT quando o Garbage Collector, são executados quando existe uma solicitação de
exibição de um trecho de código JavaScript, e estão disponíveis durante toda a execução da
JavaScript Engine, por isso estão diretamente ligados ao passo de execução dentro do
processo de interpretação de um código JavaScript.

\section{A JAVASCRIPT ENGINE SPIDERMONKEY}

Escrita utilizando linguagens de programação como C, C++ e JavaScript, pode ser
executado em vários sistemas operacionais e devices como celulares, tablets e até mesmo
em aparelho de TV, o que a torna muito adaptável e com robustez e por ser uma JavaScript
Engine, a SpiderMonkey\footnote{A JavaScript Engine SpiderMonkey é um projeto de código aberto criado por Brendan Eich, na Netscape
Communications em meados de 1996, e é considerada a primeira JavaScript Engine da história.} possui sua estrutura como uma virtual machine e segue um fluxo
de interpretação de códigos JavaScript muito próxima ao apresentado no tópico 5.1 Fluxo
de funcionamento da JavaScript Engine.
Mesmo tendo sua base de construção como sendo uma JavaScript Engine, sua estrutura
contem algumas implementações que visam melhorar o desempenho e performance na
geração e interpretação de códigos JavaScript, bem como otimizar o gerenciamento e de
uso de memória durante o período em que está em execução.
Como uma forma de melhorar o desempenho durante a execução do JavaScript, a
SpiderMonkey possui duas camadas de JITs que funcionam separadamente e com objetivos
diferentes. A primeira camada é conhecida apenas como Baseline Compiler, e que tem como
função apenas uma compilação preliminar, ou seja, sem aplicação de análise e otimização
do código recebido durante a execução, e foi introduzida para substituir a antiga camada
method JIT conhecida como JaegerMonkey, com o objetivo de facilitar a manutenção da
JavaScript Engine e por possibilitar a otimização de novas funcionalidades que podem ser
criadas nas próximas versões da linguagem JavaScript.
A segunda camada, chamada de IonMonkey, foi desenvolvida com foco em performance e
otimização na geração do código, sendo inteiramente um method JIT. Essa camada JIT
utiliza estratégias diferentes para acelerar e otimizar várias operações. As operações mais
comuns e que são relevantes para a execução do programa JavaScript são propriedades e
chamadas de função, e passam por uma otimização para obter performance. As estratégias
de otimização são classificadas em 5 categorias que são GetProperty, SetProperty,
GetElement, SetElement e Call, dentre elas cita-se como exemplo para explicação as
estratégias abaixo:

\begin{itemize}
\item GetProperty (obj.prop) – essa otimização só funciona se o objeto de argumentos é
usado de maneiras bem definidas dentro da função. A função que contém o
argumento é autorizada a utilizar o objeto argumentos.
\item Call – uma função chamada f(x) envia um frame para a pilha call, seguido de uma
chamada, o corpo da função é recebida é copiada, otimizada e devolvida para o
frame de execução.
Para que seja possível interagir com a JavaScript Engine SpiderMonkey, existe no código
fonte uma aplicação chamada de JS Shell, que nada mais é que uma aplicação de CLI que
possibilita a execução de funções que acessam diretamente instruções da JavaScript
Engine como, por exemplo, obter o resultado bytecode do processo de compilação de um
programa JavaScript, e que também é utilizado para a execução de códigos de teste de
performance de qualidade da própria SpiderMonkey
\end{itemize}

\section{A API SIMD.js}

Em 1966, M. J. Flynn desenvolveu quatro classificações na arquitetura de computadores
chamada de taxonomia de Flynn, para descrever modos distintos de entradas e saída de
dados que envolvem conceitos de paralelismo. O “Single Instruction, multiple data”,
referenciado no livro Introduction to Parallel Processing: Algorithms and Architectures, de
Behrooz Parhami(1999) como SIMD Array Processors é uma das quatro classificações da
taxonomia de Flynn conhecidas como um caso especial de um programa com um fluxo
único de entrada de instruções, múltiplos fluxos de entrada de dados e múltiplas saídas de
resultado, chamado de paralelismo de dados, e tem seu fluxo de funcionamento
exemplificado na Figura 6.

IMAGEM6

Utilizando a classificação SIMD, obtém-se um aumento de velocidade substancial em
diversas aplicações como gráficos 3D, processamento de imagens e numérico, criptografia
entre outras; e se considerarmos a aplicação da SIMD no browser, o aumento de
desempenho na utilização de tecnologias como WebGL, Canvas, ASM.js entre outras é
significativo e por utilizar poucas instruções para ser executada, a duração de bateria dos
aparelhos onde são utilizadas instruções SIMD aumenta consideravelmente.
Com foco na web, a interface de programação de aplicações SIMD.js é uma biblioteca
JavaScript que esta sendo desenvolvida por empresas como Intel, Google e Mozilla com a
intenção de inserir novos tipos e funções paralelas na linguagem JavaScript. Uma das
inclusões é referente ao tipo Float32x4, que consiste em representar 4 valores float32,
conhecido como Single-precision floating-point format, que podem ser enviados
simultaneamente para a JavaScript Engine, pois a SIMD.js possui todas as operações
aritméticas básica e operações para rearranjar, carregar e armazenar esses valores.
Na biblioteca SIMD.js, existem outros tipos implementados além do tipo Float32x4 (A
implementação do tipo Float32x4 segue a especificação 4 IEEE-754 32-bits floating point
numbers), conforme mostra a Tabela 1:

\begin{tabular}{|c|c|}
\hline
Tipo de variável & Especificação técnica do tipo\\
\hline
int32x4 & 4 32bits Signed Integers \\
\hline
float64x2 & 2 IEEE-754 64bit Float Point Numbers \\
\hline
float32x4Array & Array de float32x4 \\
\hline
int32x4Array & Array de int32x4 \\
\hline
float64x2Array & Array de float64x2 \\
\hline
\end{tabular}

Observando a hierárquica Figura 7, apresenta uma relação entre seus objetos JavaScript da
biblioteca SIMD.js o esquema se representa da forma como segue:

IMAGEM7

Utilizando uma aplicação teste feita por Peter Jensen da Intel Corporation, que implementa e
renderiza uma representação do o conjunto de Mandelbrot de fractal, percebe-se um ganho
de perfomance durante a execução de quase 50% de frames por segundo (FPS), como
mostram as Figuras 8 e 9 e os respectivos testes sem a utilização da biblioteca SIMD.js e
depois de aplicar a biblioteca JavaScript.

IMAGEM8
IMAGEM9

Na implementação do teste, quando a execução do script faz uso dos recursos de
computação paralela, foram utilizadas duas variáveis com tipos providos pelo SIMD.js, que
são SIMD.Int32x4 e SIMD.Float32x4, fazendo com que sejam executadas as operações das
funções add(), sub() e mul() simultaneamente para 4 valores informados em cada execução
das funções, como pode ser visto se acessado o código fonte do teste.

Atualmente o foco do desenvolvimento da SIMD.js é suportar tanto arquiteturas x86 com
Stream SIMD Extensions quanto arquiteturas ARM com tecnologia NEON.

\section{CONCLUSÃO}

Ao término deste projeto de iniciação científica, percebe-se que existem muitos esforços
para que cada vez mais recursos de computação paralela sejam utilização na linguagem
JavaScript e que estão partindo de empresas produtoras e consumidoras de tecnologias
web, tanto quando de desenvolvedores JavaScript.
No caso da JavaScript Engine da Mozilla, a SpiderMonkey, o caminho para a implementação
de mais recursos que possam prover funcionalidades de computação paralela já estão
caminhando bem por conta da criação de dos dois modos de compilação JIT que essa
Engine possui. No caso dos outros browsers como o Google Chrome, por exemplo, a
demostração de interesse prover esses recursos é menor, mesmo que esteja envolvida com
o projeto citado na pesquisa, o SIMD.js, e no caso da Intel, existe interesse em continuar
desenvolvendo e implementando essas novas funcionalidades por ser uma empresas que é
envolvida com recursos de hardware e que por conta disso, pode ganhar em qualidade de
performance dos web browsers que rodam na plataforma dessa empresa.
Com relação as definições da JavaScript Engine e as documentações da Engine
SpiderMonkey, observa-se que o desenvolvimento e a documentação desse projetos, sejam
publicações acadêmicas ou informais como posts em sites, ainda é muito restrito a um grupo
muito pequeno de pessoas que estão realmente envolvidas com desenvolvimento em
linguagem C++ e que tem um histórico de envolvimento no desenvolvimento de JIT
Compilers, com isso, novos desenvolvedores podem ter uma curva de adaptação e
aprendizado muito grande e com isso causar falta de interesse e com essa área.
A utilização de recursos de computação paralela como na SIMD.js traz ganhos realmente
significativos em relação a performance e utilização de recursos computacionais como por
prover melhor utilização dos recursos de processador, pode ser visto no teste feito utilizando
conjuntos de Mandelbrot. Muitos outros esforços já estão acontecendo para que mais
recursos de computação paralela sejam disponibilizados e utilizados de forma mais objetiva
e fácil para os desenvolvedores JavaScript, como por exemplo a especificação ASM.js que
esta em desenvolvimento e tem o objetivo de ser um subconjunto de funções para serem
utilizadas em JavaScript e a qual tem forte apoio do próprio criador da linguagem, Brendan
Eich.


\section{BIBLIOGRAFIA}
ACM Digital Library. A LISP garbage-collector for virtual-memory computer system.
Disponível em: \url{http://dl.acm.org/citation.cfm?id=363280}.
Acesso em: 20 de Maio de 2015.

ACM Digital Library. A parallel, real-time garbage collector.
Disponível em: \url{http://dl.acm.org/citation.cfm?id=378823}.
Acesso em: 20 de Maio de 2015.

AHO, Alfred V. et al. Compilers: Principles, Technique, and Tools, Boston, MA: Pearson Education, 2007, 1009 p.

ASM.js – Working Draft Specification.
Disponivel em: \url{http://asmjs.org/spec/latest/}.
Acesso em: 18 de Agosto de 2015.

DAHL, Ryan. NodeJS.
Disponível em: \url{https://nodejs.org/}.
Acesso em: 26 de Abril de 2015.

Demo Links with Mandelbrot tests.
Disponivel em: \url{http://peterjensen.github.io/idf2014-simd/idf2014-simd.html}.
Acesso em: 2 de Agosto de 2015.

ECMAScript® 2015 Language Specification, Geneva, CHE: Ecma Internacional, n. 262, junho. 2015.

FLANAGAN, David. JavaScript: The Definitive Guide. Sebastopol, CA: O'Reilly Media, 2011. 1096 p.

FLYNN, Michael J. Some Computer Organizations and Their Effectivness. IEEE Transactions on Computers. Vol. c-21, No.9, September 1972.
Disponível em: \url{http://www.cs.utah.edu/~kirby/classes/cs6230/Flynn.pdf}.
Acesso em: 15 de Agosto de 2015.

Intel. Parallel JavaScript.
Disponível em: \url{https://software.intel.com/enus/blogs/2011/09/15/parallel-javascript}.
Acesso em: 10 de Maio de 2015.

JavaScript AST Visualizer.
Disponível em: \url{http://jointjs.com/demos/javascript-ast}.
Acesso em: 10 de Fevereiro de 2015.

Jinks, Pete. Anatomy of a Compiler.
Disponivel em: \url{http://www.cs.man.ac.uk/~pjj/farrell/comp3.html}.
Acesso em: 03 de Março de 2015

Mozilla Developer Network. JavaScript.
Disponível em: \url{https://developer.mozilla.org/enUS/docs/Web/JavaScript}.
Acesso em: 10 de Dezembro de 2014.

Mozilla Developer Network. SpiderMonkey.
Disponível em: \url{https://developer.mozilla.org/en-US/docs/Mozilla/Projects/SpiderMonkey}.
Acesso em: 27 de Setembro de 2014.

NEON – ARM.
Disponível em: \url{http://www.arm.com/products/processors/technologies/neon.php}.
Acesso em: 13 de Agosto de 2015.

O Reilly. What Is Web 2.0: Design Patterns and Business Models for the Next Generation of Software by Tim O'Reilly.
Disponível em: \url{http://www.oreilly.com/pub/a/web2/archive/what-isweb-20.html}.
Acesso em: 29 de Junho de 2015.

PARHAMI, Behrooz. Introduction to Parallel Processing, Santa Barbara, CA: PLENUM SERIES IN COMPUTER SCIENCE, 2002, 532 p.

PIENAAR, Jacques A.; HUNDT, Robert. JSWhiz: Static Analysis for JavaScript Memory Leaks.
Disponível em: \url{http://static.googleusercontent.com/media/research.google.com/en//pubs/archive/40738.pdf}.
Acesso em: 10 de Junho de 2015

SIMD numeric type for EcmaScript.
Disponível em: \url{https://github.com/tc39/ecmascript_simd}.
Acesso em: 10 de Janeiro de 2015.

SpiderMonkey Internal.
Disponível em: \url{https://developer.mozilla.org/enUS/docs/Mozilla/Projects/SpiderMonkey/Internals}.
Acesso em: 7 de Abril de 2015.

Taligarsiel. How browsers work.
Disponível em: \url{http://taligarsiel.com/Projects/howbrowserswork1.htm}.
Acesso em: 8 de Abril de 2015.

TRATT, Laurence. Dynamically Typed Languages. United Kingdom: Advances in Computers, 2009.
Disponível em: \url{http://tratt.net/laurie/research/pubs/html/tratt__dynamically_typed_languages/}.
Acesso em: 15 de Maio de 2015.

The Baseline Compiler Has Landed.
Disponivel em: \url{https://blog.mozilla.org/javascript/2013/04/05/the-baseline-compiler-has-landed/}.
Acesso em: 10 de Junho de 2015.

Wikia. Browser War I.
Disponível em: \url{http://browserwars.wikia.com/wiki/Browser_War_I}.
Acesso em: 16 de Fevereiro de 2015.

W3C. Document Object Model (DOM).
Disponível em: \url{http://www.w3.org/DOM/#what}.
Acesso em: 7 de Janeiro de 2015.

W3C. A Short History of JavaScript.
Disponível em: \url{https://www.w3.org/community/webed/wiki/A_Short_History_of_JavaScript}.
Acesso em: 15 de Janeiro de 2015.



\section{}

Aluno: Clauber Pereira Stipkovic Halic \url{clauber.halic@gmail.com}
Orientador: Dr. Calebe de Paula Bianchini \url{calebe.bianchini@mackenzie.br}




\bibliographystyle{plainnat}
\bibliography{example}

\end{document}
